%\documentclass[notes=show,xcolor=table]{beamer}
\documentclass[xcolor=table]{beamer}
% german and utf8
\usepackage[ngerman]{babel}
\usepackage[utf8]{inputenc}

\usepackage{lmodern}
% Das Paket lmodern erspart die folgenden Warnungen:
% LaTeX Font Warning: Font shape `OT1/cmss/m/n' in size <4> not available
% (Font)              size <5> substituted on input line 22.
% LaTeX Font Warning: Size substitutions with differences
% (Font)              up to 1.0pt have occurred.

% beamer style
%\usepackage{beamerthemeshadow}
\usetheme{Warsaw}
\usecolortheme[named=blue]{structure}


\setbeamertemplate{footline}{
	\leavevmode%
	\hbox{%
		\begin{beamercolorbox}[wd=.15\paperwidth,ht=2.25ex,dp=1ex,center]{date in head/foot}%
			\usebeamerfont{date in head/foot}\insertshortdate{}
		\end{beamercolorbox}%
%		\begin{beamercolorbox}[wd=.2\paperwidth,ht=2.25ex,dp=1ex,center]{author in head/foot}%
%			\usebeamerfont{author in head/foot}\insertshortauthor%~~(\insertshortinstitute)
%		\end{beamercolorbox}%
		\begin{beamercolorbox}[wd=.7\paperwidth,ht=2.25ex,dp=1ex,center]{title in head/foot}%
			\usebeamerfont{title in head/foot}\insertshorttitle
		\end{beamercolorbox}%
		\begin{beamercolorbox}[wd=.15\paperwidth,ht=2.25ex,dp=1ex,right]{date in head/foot}%
			\usebeamerfont{date in head/foot}\insertframenumber{} / \inserttotalframenumber{}
		\end{beamercolorbox}%
	}%
	\vskip0pt%
}

% allow colored tables
% http://tex.stackexchange.com/questions/68371/how-to-highlight-table-rows-by-colors-in-beamer
% for a more sophisticated solution see http://tex.stackexchange.com/a/18430/10327
\rowcolors{1}{gray!30}{gray!10}

\makeatletter
\def\rowcolor{\noalign{\ifnum0=`}\fi\bmr@rowcolor}
\newcommand<>{\bmr@rowcolor}{%
	\alt#1%
		{\global\let\CT@do@color\CT@@do@color\@ifnextchar[\CT@rowa\CT@rowb}%
		{\ifnum0=`{\fi}\@gooble@rowcolor}%
}

\newcommand{\@gooble@rowcolor}[2][]{\@gooble@rowcolor@}
\newcommand{\@gooble@rowcolor@}[1][]{\@gooble@rowcolor@@}
\newcommand{\@gooble@rowcolor@@}[1][]{\ignorespaces}
\makeatother

\usepackage{color, colortbl}
\definecolor{LRed}{rgb}{1,.8,.8}
\definecolor{MRed}{rgb}{1,.6,.6}
\definecolor{HRed}{rgb}{1,.2,.2}

% speech bubbles
% http://tex.stackexchange.com/questions/38805/simple-speech-bubbles-arrows-or-balloon-like-shapes-in-beamer
\usepackage{tikz}
\usetikzlibrary{calc,shapes.callouts,shapes.arrows}
% usage: \arrowthis{There is}{Corot}
\newcommand{\arrowthis}[2]{
	\tikz[remember picture,baseline]{\node[anchor=base,inner sep=0,outer sep=0]%
		(#1) {\underline{#1}};
		\node[overlay,single arrow,draw=none,fill=red!50,anchor=tip,rotate=60] 
	at (#1.south) {#2};}%
}%
\newcommand{\speechthis}[2]{
	\tikz[remember picture,baseline]{\node[anchor=base,inner sep=0,outer sep=0]%
		(#1) {\underline{#1}};\node[overlay,ellipse callout,fill=blue!50] 
	at ($(#1.north)+(-.5cm,0.8cm)$) {#2};}%
}%
\newcommand{\bubblethis}[2]{
	\tikz[remember picture,baseline]{\node[anchor=base,inner sep=0,outer sep=0]%
		(#1) {\underline{#1}};\node[overlay,cloud callout,callout relative pointer={(0.2cm,-0.7cm)},%
	aspect=2.5,fill=yellow!90] at ($(#1.north)+(-0.5cm,1.6cm)$) {#2};}%
}%
\newcommand{\pointthis}[2]{
	\tikz[remember picture,baseline]{\node[anchor=base,inner sep=0,outer sep=0]%
		(#1) {\underline{#1}};\node[overlay,rectangle callout,%
	callout relative pointer={(0.2cm,0.7cm)},fill=green!50] at ($(#1.north)+(-.5cm,-1.4cm)$) {#2};}%
}%


% allow strikethrough with \sout
% http://tex.stackexchange.com/questions/23711/strikethrough-text
%\usepackage{ulem}

% multimedia
% http://mo.mathematik.uni-stuttgart.de/inhalt/aussage/aussage1325/
% http://tex.stackexchange.com/questions/429/animation-in-pdf-presentations-without-adobe-reader
% http://www.ctan.org/pkg/animate

% againframe 
% http://stackoverflow.com/questions/171708/latex-beamer-package-change-frame-title-in-againframe
% http://tex.stackexchange.com/questions/31031/beamer-repeat-variations-of-a-frame

% todo notes
% http://tex.stackexchange.com/questions/11177/how-to-write-hidden-notes-in-a-latex-file

% http://www.texample.net/tikz/examples/feature/overlays/
% http://tex.stackexchange.com/questions/14769/add-more-anchors-to-standard-tikz-nodes

%\beamersetuncovermixins{\opaqueness<1>{25}}{\opaqueness<2->{15}}
% sorgt dafuer das die Elemente die erst noch (zukuenftig) kommen nur schwach angedeutet erscheinen
% klappt auch bei Tabellen, wenn teTeX verwendet wird\ldots

% http://tex.stackexchange.com/questions/16357/how-can-i-position-an-image-in-an-arbitrary-position-in-beamer
\usepackage{tikz}
%\usetikzlibrary{calc, trees, positioning, arrows, shapes, shapes.multipart, shadows, matrix, decorations.pathreplacing, decorations.pathmorphing}
\usetikzlibrary{arrows}%, shapes}%, shapes.multipart}%, automata}
\tikzset{
  every boverlay node/.style={
    draw=black,fill=white,rounded corners,anchor=north west,
  },
}
\tikzset{
  every toverlay node/.style={
	anchor=center
  },
}
\tikzset{
  every overlay node/.style={
    draw=white,fill=white,rounded corners,anchor=north west,
  },
}
% Usage:
% \tikzoverlay at (-1cm,-5cm) {content};%
% or
% \tikzoverlay[text width=5cm] at (-1cm,-5cm) {content};%
\def\tikzboverlay{% A bordered overlay
   \tikz[baseline,overlay]\node[every boverlay node]
}%
\def\tikztoverlay{% A transparent overlay
   \tikz[baseline,overlay]\node[every toverlay node]
}%
\def\tikzoverlay{% A non-transparent unbordered overlay
   \tikz[baseline,overlay]\node[every overlay node]
}%



\title{Optimierung von Sensornetzen}
\author{Daniel Kniese\and\\Mitja Richter}
\date{1. März 2013}

\begin{document}

	\begin{frame}
		\titlepage
	\end{frame}

	\begin{frame}
		\frametitle{Inhaltsverzeichnis}
		\tableofcontents
	\end{frame}


	\AtBeginSection[]{
		\begin{frame}
			\frametitle{Inhaltsverzeichnis}
			\tableofcontents[hideothersubsections]
		\end{frame}
	}
	
	\section{Sensornetze}
	\subsection{statische Netze}
		\begin{frame}
			\frametitle{Anwendungen statischer Netze}
			\tikztoverlay at (5.35cm,-0.44cm) (atom) {
			\includegraphics[scale=0.3]{bilder/atomwaffen.png}
		};%
		\note{Organisation des Vertrags über das umfassende Verbot von Nuklearversuchen(CTBTO) 50 primäre und 120 sekundäre seismologische Messstationen, die seismische Aktivitäten aufzeichnet und versucht nukleare Explosionen von Erdbeben zu unterscheiden}
		\end{frame}
		\begin{frame}
			\frametitle{Vorteile statischer Sensornetze}
		\end{frame}
		
	\subsection{mobile Netze}
		\begin{frame}
			\frametitle{Anwendungen mobiler Sensornetze}
			\begin{itemize}
				\item militärische Überwachung (Beispiele)
				\item Überwachung des Ökosystems (Beispiele)
			\end{itemize}
		\end{frame}
		\begin{frame}
			\frametitle{Vorteile mobiler Netze}
			Vorteile:			
			\begin{itemize}
			\item großer Einsatzbereich
			\item skalierbar
			\item flexibel
			\item Infrastruktur
			\end{itemize}
			\note{können sich gegebenheiten anpassen, infrastruktur muss nicht gewartet werden}
		\end{frame}
	\section{Problemaufbau}
	\subsection{Einsatzgebiet}
		\begin{frame}
			\frametitle{Einsatzgebiet}
			\includegraphics[scale=0.48]{bilder/aufbau1.png}
			\note{wir haben eine beliebige Anzahl von Robotern mit entsprechenden Sensoren ausgestattet. diese werden jetzt zufällig in einer abgeschlossenen konvexen Umgebung eingesetzt.}
		\end{frame}
		\begin{frame}
			\frametitle{Sensorfunktion}
			\includegraphics[scale=0.48]{bilder/sensor1.png}
			\tikzoverlay at (-0.4cm,5cm) {\small $\phi(x,y)=a\cdot exp(-b(x-y)^2)$};
			\tikzoverlay at (-0.4cm,4.5cm) {\small mit $a=1000$ und $b=-10$};
		\note{keine Linearkombination von Standardnormalverteilungen, kann bedeuten hohe ereignisdichte in dieser region}
		\end{frame}
		
		\begin{frame}
			\frametitle{Sensorfunktion}
			\tikztoverlay at (2.35cm,-0.44cm) (aufbau) {
			\includegraphics[scale=0.35]{bilder/aufbau1.png}};
			\tikztoverlay at (8cm,-0.44cm) (sensor) {
			\includegraphics[scale=0.35]{bilder/sensor1.png}};
		\end{frame}

 
		\subsection{Vonoroi-Partitionen}		
		
		\begin{frame}
			\frametitle{Aufteilung}
			\includegraphics[scale=0.5]{bilder/vonoroiFische.png}
			\note{Es gibt eine afrikanische Buntbarschart, die ihr territoriales Verhalten ausüben, indem sie untereinander das Gebiet in sich nicht überschneidende Teilgebiete unterteilen. Sie bilden kleine Gruben deren Ränder dann ein Polygonmuster bilden. Mann kennt dieses Muster auch unter dem Namen Vonoroipartitionen.}
		\end{frame}
		
		\begin{frame}
			\frametitle{Vonoroi-Partitionen}
			Gegeben sei:
			\begin{itemize}
				\item[•] Menge $S\subset \mathbb{R}^2$
				\item[•] Menge $P=\{p_1,p_2,...,p_n\} \subset S$
			\end{itemize}
			Dann sei die Menge der Vonoroi-Partitionen $\{V_1(P),V_2(P),...,V_n(p)\}$ gegeben durch:
			\begin{itemize}
				\item[•] $V_i(P)=\{q \in S| \|q-p_i\|\leq \|q-p_j \|, \forall p_j \in P\}$
			\end{itemize}
		\end{frame}
		\begin{frame}
			\frametitle{Vonoroi-Partitionen}
			\tikztoverlay at (5.35cm,-0.44cm) (atom) {
			\includegraphics[scale=0.6]{bilder/vonoroi.png}
		};%
		\end{frame}
		\begin{frame}
			\frametitle{Vonoroi-Partitionen}
			\includegraphics[scale=0.48]{bilder/aufbau2.png}
		\end{frame}
		\subsection{Aufgabenstellung}
		\begin{frame}
			\frametitle{Was ist das Ziel?}
			\includegraphics[scale=0.35]{bilder/sensor2.png}
		\end{frame}
		\begin{frame}
			\frametitle{Was ist das Ziel}
			\tikztoverlay at (2.35cm,-0.44cm) (endzustand) {
			\includegraphics[scale=0.37]{bilder/endzustand.png}};
			\tikztoverlay at (8.3cm,-0.44cm) (sensor) {
			\includegraphics[scale=0.37]{bilder/sensor1.png}};
			\note{Wir wollen eine gute Abdeckung mit guter Verlässlichkeit}
		\end{frame}
		\begin{frame}
			\frametitle{Wie wird dieses Ziel formalisiert}
			\begin{itemize}
			\item Unverlässlichkeit der Sensorinformationen eines Roboters mit einer quadratischen Funktion $f(x)$:\\
			$f(\| q-p_i \|)=\frac{1}{2}(\| q-p_i \|)^2$
			\note{Man erkennt, dass je weiter der Roboter vom Punkt q entfernt ist, desto unverlässlicher werden die Daten}
			\item Systemperformanz wird in Abdeckungsfunktion gemessen:\\
			$H(p_1,...,p_n)=\sum^N_{i=1}\int_{V_i}\frac{1}{2}(\| q-p_i \|)^2 \phi(q)dq$\\
			\item Minimierung $\rightarrow$ Optimalität
			\end{itemize}
			\note{Mann sieht dass hohe unverlässlichkeit teuer ist und hohe werte in der sensorfunktion auch. unverlässlichkeit ist schlimmer in guten spots als in schlechten}
		\end{frame}
	\begin{frame}
		\frametitle{Vergleich Vonoroi-Partition und Festkörper}
		\includegraphics[scale=0.6]{bilder/stein1.png}\quad
		\includegraphics[scale=0.3]{bilder/stein2.jpg}\\
		\tikzoverlay at (-0.7cm,0cm) {Schwerpunkt über Sensorfunktion};		
		\tikzoverlay at (4.7cm,2.3cm) {\huge $\rightarrow$};		
		\tikzoverlay at (4.6cm,0cm) {\huge $\rightarrow$};			
		\tikzoverlay at (5.3cm,0cm) {Schwerpunkt über Dichtefunktion};		
		
	\end{frame}
	\begin{frame}
		\frametitle{Festkörpereigenschaften}
		Bestimmte Eigenschaften können auch auf Vonoroi-Partitionen übertragen werden:
			\begin{itemize}
				\item Massemoment:\\
				$M_{V_i}=\int_{V_i}\phi(q)dq$
				\item Erstes Moment:\\
				$L_{V_i}=\int_{V_i}q\phi(q)dq$
				\item Schwerpunkt:\\
				$C_{V_i}=\frac{M_{V_i}}{L_{V_i}}$
			\end{itemize}
	\end{frame}
	\begin{frame}
		Vereinfachung für Optimierung:
		\begin{itemize}
			\frametitle{Optimierung}
			\item $\frac{\delta H}{\delta p_i}=-\int_{V_i}(q-p_i)\phi (q)dq=-M_{V_i}(C_{V_i}-L_{V_i})$
		\end{itemize}
		$\rightarrow$ Optimalität wenn Roboter in jeweiligen Schwerpunkten der Vonoroi-Partitionen
		\begin{itemize}
			\item Finden globaler Minima ist NP-Hart $\rightarrow$ Beschränkung auf lokale Minima 
		\end{itemize}
	\end{frame}
	\begin{frame}
		\frametitle{Wie bewegt sich der Roboter?}
		Bewegung des Roboters modelliert durch:
		\begin{itemize}
			\item \.{p}$ =u_i$
		\end{itemize}
		wobei das control law $u_i$ gegeben ist durch:
		\begin{itemize}
			\item $u_i=k_i(p_i-C_{V_i}$
		\end{itemize}
		$p_i$ macht Gradientenabstieg
		\tikztoverlay at (4.5cm,2cm) (sensor) {
			\includegraphics[scale=0.35]{bilder/gradient.jpg}};
	\end{frame}
	\subsection{Annahmen}
	\begin{frame}
		\frametitle{Annahmen}
		\begin{itemize}
			\item Sensorfunktion bekannt
			\item Sensorfunktion konstant
			\item Roboter kann eigene Vonoroi-Partition berechnen
		\end{itemize}
	\end{frame}
	\subsection{Algorithmus}
	\begin{frame}
		\frametitle{Algorithmus zur Optimierung von $h$}
		Loop:		
		\begin{itemize}				
			\item Abstand zu anderen Robotern berechnen
			\item eigene Vonoroi-Partition berechnen
			\item den Schwerpunkt $C_{V_i}$ berechnen
			\item control-input $u_i=k_i(p_i-C_{V_i}$ anwenden
		\end{itemize}
	\end{frame}
	\section{Wofür brauchen wir KI?}
	\subsection{Annahmen}
	\begin{frame}
		\frametitle{Annahmen}
		\begin{itemize}
			\item Sensorfunktion \textbf{nicht} bekannt
			\item Robotor kann Funktionswert an seiner Stelle messen
			\item Sensorfunktion konstant
			\item Roboter kann eigene Vonoroi-Partition berechnen
		\end{itemize}
	\end{frame}
	\subsection{Neuronale Netze}
	\begin{frame}
		\frametitle{Neuronales Netzwerk zur Abschätzung von $H$}
		\includegraphics[scale=0.35]{bilder/nn.png}
	\end{frame}
	\section{Ergebnisse}
	\section{Fazit}
	
	
\end{document}
